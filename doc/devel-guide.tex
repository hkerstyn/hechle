\documentclass{article}
\title{\lazydog  Development Guide}
\author{Kerstin Hoffmann}
\date{}
\usepackage{listings}
\usepackage{xspace}
\usepackage[hidelinks]{hyperref}

\setlength{\parindent}{0pt}
\renewcommand\v[1]{\texttt{#1}}
\newcommand\lazydog{\v{lazy-dog}\xspace}
\def\focus{0}
\newcommand\note[1]{{\it Note:} {#1}}
\newcounter{rememberEnumi}

\newcommand\tclass[1]{$\langle${\it{#1}}$\rangle$}
\def\tcom{\tclass{comment}\xspace}
\def\tident{\tclass{identifier}\xspace}
\def\ttrait{\tclass{trait name}\xspace}
\def\tsym{\tclass{symbol}\xspace}
\def\tuint{\tclass{unsigned integer literal}\xspace}
\def\tint{\tclass{integer literal}\xspace}
\def\tfl{\tclass{floating literal}\xspace}
\def\tstr{\tclass{string literal}\xspace}
\def\tinv{\tclass{invalid}\xspace}

\begin{document}
% if focus is 1, everything until \fi will be ignored
% because each recompile sets the pdf viewer to first page

\if0\focus
\maketitle
\tableofcontents

\section{Overview}
\subsection{This document}
This document is the Development Guide
for the programming language \lazydog.\\

This is a WIP {\bf Documentation} of the language
that will aide the development of \lazydog's compiler.
Is is eventually to be joined by a proper
documentation.

It does not however document the implementation
architecture or usage of the \lazydog
compiler.\\

This document is {\bf not to be read in order},
as each language aspect is discussed in all chapters.

The {\bf first chapter}
\footnote{A chapter refers to a \v{\textbackslash section} in \LaTeX}
is devoted to laying
out general information regarding this document
and the language as a whole.\\

The {\bf second chapter} precisely defines
the language's grammar and introduces terms
regarding the syntax. 
It does not however explain the purpose of any language
feature.\\

The {\bf third chapter} explains intuitively the way the language
works by outlining the general ideas and sketching rough
syntax examples.\\

The {\bf fourth chapter} defines precisely what each
aspect of the language does.\\

The {\bf last chapter} delves into how the language is
to be translated into C.\\

Note the absence of any chapter devoted to the compiler itself,
as stated above.


\subsection{Design Philosophy}
The design of \lazydog has the following primary goals:
\begin{enumerate}
    \item {\bf Ease of Compilation:}
    To ensure that \lazydog will eventually see the light
    of day, it is paramount to design the language in such
    a manner that it can easily and efficiently be compiled
    to C.
    \begin{itemize}
        \item no excessive compile time disambiguation
        \item no excessive compile time evaluation
        \item a simple memory model
        \item no purely functional language design
    \end{itemize}
    \item {\bf Liberal Consistency:}
    \lazydog should refrain from overly adhering to any
    specific design idea. 
    Consistency should always be sacrificed to serve any
    of the other goals.
    Given that, consistency should still be generally strived for
    \item {\bf Ease of Typing:} \lazydog code should be
    reasonably easy to type with a German quertz keyboard.
    \begin{itemize}
        \item banning of the backtick, it is the worst character
        \item identifiers in \v{dash-case}
        \item a general preference for words over special characters
    \end{itemize}
    \item {\bf Aesthetics:} \lazydog should look sleek.
    We want all the syntax sugar.
    \setcounter{rememberEnumi}{\theenumi}
\end{enumerate}

\begin{samepage}
    
The following secondary goals, while beneficial, are of lesser concern.
\begin{enumerate}
    \setcounter{enumi}{\therememberEnumi}
    \item {\bf Simplicity:} simplicity is intrinsically valuable,
    however the language should not be simplified to the point
    where it introduces unproductive abstraction
    \item {\bf Performance:} compilation and execution of a
    \lazydog program should be fast enough not to render
    the language unusable. Performance should not however
    make the language overly convoluted.
    \item {\bf Legibilty:} code in \lazydog should
    be reasonably easy to read and understand.
    This should not be achieved through {\bf verbosity}.
    \item {\bf Ease of Learning:} \lazydog should be somewhat
    easy to pick up. 
    This is can be improved by quality documentation
    and by sticking to conventions laid out by other languages.
\end{enumerate}
\end{samepage}


\section{Syntax}
At present, the documentation in this chapter assumes
a {\bf single source file}.

The source file is in ASCII. It is planned to eventually add Unicode
support, which will enable non-ASCII characters in comments and string literals.

\subsection{Tokenization}
A {\bf Token} is a substring of the source file.
It is aware of 
\begin{itemize}
    \item its location in the file
    \item the corresponding string data
    \item its {\bf classification}
\end{itemize}

Tokens are classified into the following token classes:
\begin{enumerate}
    \item \tcom
    \item \tident
    \item \ttrait
    \item \tsym
    \item \tint
    \item \tfl
    \item \tstr
    \item \tinv
\end{enumerate}
The classification does not provide any additional data.

The {\bf Tokenizer} splits the source file into tokens
and classifies them. The algorithm is like so:
\begin{enumerate}
    \item exit at eof
    \item discard any whitespace
    \item find the longest string starting at that position
    that still fulfills the requirement
    of at least one token class
    \item the token classes are created such that there
    will be exactly one matching class
    \item once such a string was found, return it with
    its unique class. go back to 1.
\end{enumerate}
Here, whitespace specifically refers to
\begin{itemize}
    \item the space character \verb$\s$
    \item the newline character \verb$\n$
    \item the tab character \verb$\t$
\end{itemize}
\fi

\subsubsection{Comments}
A \tcom is any string that
\begin{enumerate}
    \item starts with the hash character \v{\#}
    \item followed by a non-hash character
    \item and does not contain a newline \verb$\n$
\end{enumerate}

Alternatively, a (multiline/inline) comment is a string that
starts and ends with at least two hashes \v{\#}.
If the end contains three or more hashes, the remainder of the line
is also part of the comment.
As such, the following are comments:
\begin{verbatim}
# single line comment

code ## inline-comment ## more-code

# multi
# line
# comment

###
multi
line 
comment
### (also part of the comment)

##
multi
line
comment
## code

\end{verbatim}

\subsubsection{Identifiers}
An \tident is a string consisting of
\begin{itemize}
    \item lowercase letters \v a-\v z
    \item digits \v0-\v9
    \item the dash \v-
\end{itemize}
such that
\begin{enumerate}
    \item it starts with a lowercase letter
    \item the last character is not a dash
    \item there are no two consecutive dashes
\end{enumerate}
Examples of identifiers (separated by whitespace):
\begin{verbatim}
    hello-world123 coffee2go x yo-what-up call-9-1-1
\end{verbatim}
Examples that are not identifiers:
\begin{verbatim}
    camelCase snake_case a--b -x x- 2nd-amendment
\end{verbatim}

\subsubsection{Trait Names}
A \ttrait is a string consisting of
\begin{itemize}
    \item lowercase letters \v a-\v z
    \item uppercase letters \v A-\v Z
    \item digits \v0-\v9
\end{itemize}
such that
\begin{enumerate}
    \item it starts with an uppercase letter
\end{enumerate}
Examples of trait names:
\begin{verbatim}
    UpperCamelCase SHOUTING Vec2D Number3 
\end{verbatim}
Examples that are not trait names:
\begin{verbatim}
    lowerCamelCase 2ndAmendment
\end{verbatim}
\subsubsection{Symbols}
A \tsym is a string consisting of the following
characters:
\begin{verbatim}
    ! $ % & ( ) * + , - . / : ; < = > ? [ ] \ _ { } | ~
\end{verbatim}
It must not be the concatenation of two or more
elementary symbols.
The elementary symbols are precisely the following:
\begin{verbatim}
    & &! &? ( ) , -> . .. ... : :: = => > [ ] _ __ { } |
\end{verbatim}
Elementary symbols that are themselves concatenations of
other elementary symbols are an exception to this rule.

A trailing dash is considered elementary.

\subsubsection{Integer Literals}
An \tint is a string consisting of
\begin{itemize}
    \item digits \v0-\v9
    \item the dash \v-
\end{itemize}
such that
\begin{enumerate}
    \item it contains no dashes or a single dash
    at the start
\end{enumerate}
Note that an integer literal may be of arbitrary length,
this is because during tokenization it is not 
converted to an integer.
Furthermore, leading zeroes are tolerated.
Examples of integer literals:
\begin{verbatim}
    -123 0 1 2 0020 99999999999999999999999999999999999999999999
\end{verbatim}
An integer literal must not be followed by \v e or \v.

\subsubsection{Floating Literals}
A \tfl is a string of any of the following patterns:
\begin{quote}
    \tint\v.\tuint\\
    \tint\v.\tuint e\tint\\
    \tint\v e\tint
\end{quote}
Note that the whitespace in above patterns is 
a display error. A \tfl does not contain whitespace.
A floating literal must not be followed by another \v e or \v.

An \tuint is an \tint not containing a dash.

Examples of \tfl:
\begin{verbatim}
    3.14 10e99999 -1.75e-14 0.0
\end{verbatim}

\subsubsection{String Literals}
A \tstr is any string
\begin{itemize}
    \item that starts and ends with \v"
    \item whose second to last character is not \verb$\$
    \item where all other occurences of \v" are preceded by \verb$\$
\end{itemize}
\subsubsection{Invalid Tokens}
An \tinv token is any string that
\begin{itemize}
    \item is not in any token class above
    \item does not contain whitespace as defined above
\end{itemize}

\section{Concepts}
\section{Language Logic}
\section{Translation to C}
\end{document}
